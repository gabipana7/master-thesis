\chapter{Models that show SOC behaviour}
\label{chap:quakeModels}
In this chapter a few models are presented in order to better understand Self-Organized Criticality and to introduce the reader into the space of seismic models.

%-------------------------------------------------------------
\section{Flyvbjerg's Simplified Sandpile Model}
In his work \cite{simplestSOC}, Flyvbjerg illustrates, as the name suggests, the simplest possible Self-Organized Critical system. His model is a simplification of the usual sandpile model, which we will describe in detail in the next section, which he calls the "random neighbor model".\par 
Consider N dynamical sites each capable of containing $z_{max}$ grains of sand. The simplest case, each can contain 1 or 0 sand grains, so $z_{max}=2$. With discrete time, at each step a random site is chosen and a sand grain is dropped on it. If it already contained a grain, the site topples and the grains redistribute to neighbors. These neighbours can be of two types: normal $N$, or absorbing $M$:
\begin{itemize}
	\item $N=N_0+N_1$ = total number of sites; $N_0$ = unoccupied, $N_1$ = occupied
	\item $M$ = absorbing sites, having the purpose of eliminating sand from the model, i.e. keeping it's dissipative characteristic.
\end{itemize}  \par 

When a topple happens there are 3 outcomes:
\begin{enumerate}
	\item Grain falls on empy site with probability $N_0/(N+M)$.
	\item Grain falls on occupied site with probability $N_1/(N+M)$, resulting in another toppling.
	\item Grain falls on absorbing site with probability $M/(N+M)$, exiting the system.
\end{enumerate}\par

The phase space of this system is thus fully characterized by 2 integer variables, i.e. having only 2 degrees of freedom:
\begin{enumerate}
	\item $N_1(\tau)$ = the number of particles in the sandpile = the number of sites occupied, since at $z_{max}=2$, the site topples.
	\item $n(\tau)$ = the number of sand particles in the avalanche.
\end{enumerate}\par 
Flyvbjerg proceeds to show that in mean field approximation, $N \to \infty$, $M \to \infty$ and $M/N \to \infty$, the model is gradually driven to a state which is critical, indicated by the absence of a characteristic scale in avalanche sizes. Furthermore, if you consider the number of occupied sites $N_1 = N/2$, it acts as an attractor point for the medium's dynamics up to errors and fluctuations of order $\sqrt{N}$. The fluctuations are crucial for the self-organizing characteristic of the model.\par 
By describing the evolution of the system using the master equation involving probability
\begin{equation}
\begin{split}
P(N_1,n;\tau+1) =& \frac{N_1+1}{N+M}P(N_1+1,n-1;\tau) \\
&+ \frac{N-N_1+1}{N+M}P(N_1-1,n+1;\tau)\\
&+ \frac{M}{N+M}P(N_1,n+1;\tau).
\end{split}
\end{equation}
By introducing scaling variables $x$, $y$ and $t$ and scaling function $f(x,y;t)$:
\begin{align}
&x = (N_1-N/2)/\sqrt{N},\\
&y = n/\sqrt{N},\\
&t = \tau/N,\\
&\mu = M/ \sqrt{N},\\
&f(x,y;t) = NP(N_1,n;\tau).
\end{align}
We get the equation for the probability density $f(x,y;t)$ that the medium is in state $x$ and the avalache has size $y$ at time $t$:
\begin{equation}
\partial_t f = [\underbrace{\frac{1}{2}(\partial_x - \partial_y)^2}_\text{diffusive term} + \underbrace{2(\partial_x - \partial_y)x}_\text{describes SOC} + \underbrace{\mu \partial_y}_\text{dispersive term}]f.
\end{equation}
Using these variable transformations to reach the equation above, it is shown that an increase of x by infinitesimal amount $dx$ initiates an avalanche in $y = 2/\sqrt{N}$ with probability $\sqrt{N}/2dx$. \par 
Also the distribution $p(x)$, i.e the state x of the system between avalanches changes with each step and as expected through the central limit theorem is a normal distribution. However, the mean is not zero, showing that the mean field result for the average state $N_1 = N/2$ has a positive correction $\sqrt{N}$. \par
The underlying conclusion of this model is that Flyvbjerg proves that a system with only two independent degrees of freedom can be SOC and that there isn't a simpler possible SOC.


\clearpage
%-------------------------------------------------------------------------
\section{The Sandpile Model}

\subsection{Bak, Tang and Wiesenfeld's Explanation of 1/f Noise}
In their work \cite{BTW}, Bak, Tang and Wiesenfeld (BTW) demonstrate that dynamical systems with spatial degrees of freedom naturally evole into a self-organized critical point. This analysis was fueled by two problems which were unexplained before:
\begin{itemize}
	\item the anomalous flicker noise, or "$1/f$" noise found for transport in systems like resistors, flow of rivers, luminosity of stars. The low-frequency power spectra of these systems display a very robust power-law behaviour.
	\item the self-similiar fractal structures of spatially extended objects, like cosmic strings, coastal lines, mountain ranges. Again, a power-law is detected in the analysis of spatio-temporal correlations extending over large periods.
\end{itemize}
In self-organization, criticality loses it's meaning from phase transitions, where a parameter (temperature for example) needs to be tuned so that the system reaches the critical point; it now represents an attractor whose scaling properties are insensitive to such parameter changes. The system, starting far from equillibrium arranges itself into a state that is barely stable.\par 

\begin{figure}[!h]
  \centering
  \includegraphics[width=.5\linewidth]{SOC_sandpile}
  \caption{An illustration from "How Nature Works" \cite{natureworks}, a drawing by Elaine Wiesendfeld in which the dropping of grains of sand on a little pile on the beach is pictured.}
  \label{fig:driverBlock}
\end{figure}


The model is described as a two-dimensional sandpile, a cellular automaton comprising of interactions of an integer variable $z$ (number of sand grains in a site) with it's nearest neighbours. A grid of sites is created, with boundary conditions $z=0$. The sites are updated at random by dropping sand on them, and when $z>K$ (a threshold value) redistribution occurs:
\begin{align}
&z(x,y) \to z(x,y)-4, \\
&z(x\pm 1,y) \to z(x \pm 1,y)+1, \\
&z(x,y\pm 1) \to z(x,y\pm 1)+1.
\end{align}
The site is initiated with random values for each site $z>>K$ and it evolves until it stops $z<K$. Simulations show that the distribution of cluster sizes and time scales follow power-law distributions.\par 
The self-organized critical state of minimally stable clusters is very robust on all length scales and they ceate fluctuations on all time scales.



%-------------------------------------------------------------------------
\subsection{Bak's Bureaucrats Model}
In his famous book, How Nature Works \cite{natureworks}, Per Bak describes a re-imagination of the sandpile model, called the bureaucrats model. The principle is similar and it is illustrated in the following picture:

\begin{figure}[!h]
  \centering
  \includegraphics[width=.5\linewidth]{SOC_bureaucrats}
  \caption{The "Office" version of the sandpile model. At each timestep a document is placed on the desk of a bureaucrat. When he finds four or more documents on his desk, he redistributes them, one to each of his neighbours, or he throws it out the window if he is at the edge of the room (the boundary conditions).}
  \label{fig:bureaucrats}
\end{figure}


%-------------------------------------------------------------------------
\section{Olami-Feder-Christensen Slider-Blocks Model }

In their paper \cite{OFC}, Olami, Feder and Christensen (OFC) argue that earthquakes are probably the most relevant paradigm of SOC. One of the first realisations in this space is the Gutenberg-Richter law which states that the rate of occurence of earthquakes of magnitude $M$ greater than $m$ is given by:
\begin{equation}
log_{10} N(M>m) = a- bm.
\end{equation}
with the parameter b having a wide range of values for different faults.
The energy released during the earthquake is thought to increase exponentially with the magnitude:
\begin{equation}
log_{10}E = c +dm
\end{equation}
Thus, the law transforms into a power law for the number of observed earthquakes with energy greater than $E$:
\begin{equation}
N(E_0>E) \approx E^{-b/d} = E^{-B}
\end{equation}
OFC devised a 2D generalization of the Burridge-Knopoff \cite{burridge} 1-dimensional spring-block model for earthquakes. Their model have some interesting qualities:
\begin{itemize}
	\item displays robust SOC behaviour over a number of conservation levels;
	\item the amount of conservation impacts the obtained power-laws;
	\item as conservation increases, the behaviour transitions from localized to nonlocalized;
	\item this conservation dependence explains the variance of the parameter in the Gutenberg-Richter law.
\end{itemize}
The model is constructed as follows :
\begin{enumerate}
	\item consider a 2D lattice of blocks;
	\item each block is connected with springs to its four nearest nighbours and to a rigid moving plate that connects all the blocks;
	\item each block interacts frictionally to a fixed plate that they move on;
	\item blocks are driven by the continuous displacement of the moving plate;
	\item when the force of one of the blocks reaches a threshold value $F_{th}$, the block slips, redefining the forces of it's neighbours;
	\item the slip of one block may result in an avalanche of slips by the nearest blocks resulting in a chain reaction.
\end{enumerate}

\begin{figure}[!h]
  \centering
  \includegraphics[width=.5\linewidth]{SOC_sliderBlocks}
  \caption{Slider Blocks Model: identical blocks of mass $m$ are linked with each other by identical springs with elastic constant $k_c$ and to a moving plate in constant motion with velocity $v$ by springs with elastic constant $k_L$. Blocks are in contact with a fixed plate, with friction.}
  \label{fig:sliderBlock}
\end{figure}

The total force exerted by the springs on a block $(i,j)$ is:
\begin{align}
F_{i,j} &= K_1[2dx_{i,j}-dx_{i-1,j}-dx_{i+1,j}]\\
&+K_2[2dx_{i,j}-dx_{i,j-1}-dx_{i,j+1}]+K_Ldx_{i,j}
\end{align}

When the plates move relative to eachother and the force exerted on a block reaches the threshold, it redistributes as follows:
\begin{align}
&F_{i \pm 1,j} \to F_{i \pm 1,j} + \delta F_{i \pm 1,j},\\
&F_{i,j \pm 1} \to F_{i,j \pm 1} + \delta F_{i,j \pm 1},\\
&F_{i,j} \to 0,
\end{align}
with the increases in forces given by:
\begin{align}
& \delta F_{i \pm 1,j} = \frac{K_1}{2K_1+2K_2+K_L}F_{i,j} = \alpha_1 F_{i,j}, \\
& \delta F_{i,j \pm 1} = \frac{K_2}{2K_1+2K_2+K_L}F_{i,j} = \alpha_2 F_{i,j}.
\end{align}
For simplification, the isotropic case is considered: $K_1 = K_2 = k_c$, subsequently the elastic ratios become $\alpha_1 = \alpha_2 = \alpha = \frac{k_c}{4k_c+k_L}$.
The model evolves as follows:
\begin{enumerate}
	\item initialize all sites' forces to a random value between 0 and 1;
	\item check if any $F_{ij} \geq F_{th}$;
	\item if yes, redistribute the force according to:
	\begin{align}
	&F_{n,n} \to F_{n,n} + \alpha F_{i,j},\\
	&F_{i,j} \to 0.
	\end{align}
	\item repeat from step 2 until the earthquake fully evolves;
	\item locate the block with $F_{max}$ and add $F_{th}-F_{max}$ to all blocks and return to the second step.
\end{enumerate}
The probability distribution of the total number of relaxations of the earthquakes is measured. This quantitiy is proportional to the energy release during an earthquake. This cellular automaton model is found to show SOC behaviour for a large number of $\alpha$ values.\par 





