\chapter{Complex Systems}
\label{chap:complexSystems}

%--------------------------------------------------------
\section{What makes a System Complex}
{\bf Complexity}, together with {\bf diversity}, are two properties that coexist when dealing with a complex system and a first step to understanding such a type of system would be to understand and differentiate between these two concepts \cite{diverscomplex}.\par

{\bf Diversity} applies to a collection of entities or a population; it requires a multitude of objects that are to be analyzed. For example, cities, ecosystems, molecular matter, plasma are diverse. When speaking of diversity, we can mean a few different characteristics of a population. For example you can mean a {\it variation} of some attribute, such as difference of height in a group of people. We can also talk about {\it diversity} of types, such as different ethnicities in a social group. Finally we can talk about differences in {\it configuration} such as different arrangments in housing units of a group of people.\par

{\bf Complexity}, on the other hand, proves to be a much more difficult concept to define. Complexity can be thought of as particular structures and patterns that cannot be easily described or predicted. A system becomes complex when {\it diverse} rule-following entities behave in an {\it interdependent} way. These entities interact over a {\it contact structure} or {\it network}. Furthermore, these entities can also {\it adapt}. For example, in a social system, individuals can learn, or in an ecosystem natural selection can take place. Another characteristic of these systems is robustness, meaning that they exhibit a certain behaviour at any spatio-temporal scale you can consider and also they can produce $large$ events, which is a concept that we will tackle at large in the following. Finally, they can attain equillibrium states, fixed points or patterns and they can produce long random sequences (chaotic behaviour).\par

As an interdisciplinary domain, complex systems draws contributions from many different fields, such as the study of {\bf self-organization} and {\bf critical phenomena} from physics, that of spontaneous order from the social sciences, chaos from mathematics, adaptation from biology, and many others. "Complex systems" is therefore often used as a broad term encompassing a research approach to problems in many diverse disciplines, including statistical physics, information theory, nonlinear dynamics, anthropology, computer science, meteorology, sociology, economics, psychology, and biology.\par 

In our studies, two concepts that we need to be familiarized with are {\it criticality} and {\it self-organization} and finally, we can combine these two under the umbrella of {\bf Self-Organization Criticality}, a concept which we use to theorise and try to explain the driving mechanism of seismic activity, the ultimate goal of this review.

%--------------------------------------------------------
\section{What is a Critical System}

\subsection{Critical phenomena}
The term {\bf critical phenomena} refers to peculiar behaviour of a system when it is near or at the point of a continuous-phase transition, also called a {\it critical} point \cite{NAP628}. A continuous-phase transition can be defined as a point at which the system changes from one state to another without a jump or discontinuity in it's properties such as internal energy, density or magnetization. In contrast to the critical point (or interchangeable, the continous-phase transition) we can talk about the familiar case of the first-order phase transition where the properties jump discontinously as a parameter of the system (pressure, or temperature for example) passes through the transition point.\par 

The peculiarities of the critical point arise because there are  certain degrees of freedom of the system that show anomalously large fluctuations on a long spatio-temporal scale, compared to a normal system far from criticality. These large fluctuations cause a breakdown of the usual macroscopic laws that characterize the system in dramatic or subtle ways and it is very challenging to learn what are the new special laws that describe the system at it's critical points.\par 

The physicist's attraction to the study of these peculiarities arise both in {\it theorists}, whom struggle with the computationally intensive interconnected laws that describe each interacting part of the system and in {\it experimentalists} whose challenge lies in making measurements close to the critical point, which requires extremely precise control of the paramater of the system (temperature, for example in condensed matter) which is to be modified in order to attain said point. But as it is with the minds of physicists, where there is complexity, there is also the fascination and drive to understand it, so this is the moving factor in the interest of studying these behaviours. The understanding gained has proven useful in a plethora of other systems, including quantum field theory in elementary-particle systems, analyses of phenomena in long polymer chains, description of percolation in macroscopically inhomogenous systems.

\subsection{Examples of phase transitions and critical points}
The following examples help us differentiate between first-order transition and a continuous transition (or critical point).\par
\paragraph{First-order transition:}
the boiling transition from liquid to vapor is a first-order transition: when the water at $1$ atmosphere pressure boils at a temperature of $100 ^{\circ}$ Celsius there is a sudden decrease in density by a factor of $1700$.

\paragraph{Continuous transition:}
the Curie point of a ferromagnetic substance is a critical point. At temperatures below $T_C$, a sample of iron has a net magnetization $\bm{M}$ that points along one of several directions , in the absence of an external magnetic field. The magnetization strength decreases with increasing temperature until $T_C$ is reached. Above this temperature, the magnetization is zero and the material is in a paramagnetic state. So we can say that at $T_C$ there is a continuous tranisition, or a critical point.

In order to facilitate the comparison between phase transitions it is convenient to introduce the concept of {\it order parameter}. For the ferromagnet, this would be the magnetization $\bm{M}$ which measures the amount of broken symmetry in the system. When there is no broken symmetry, the order parameter is a quantity sensitive to the difference between the two phases, below or above the transition temperature. For the boiling water example, the order parameter would be the actual density of the fluid.


%--------------------------------------------------------
\section{Self-Organized Criticality}

\subsection{Self-Organization}
Self-organization is the process by which individual components of a system organize their communal behaviour to create global order by interactions amongst themselves rather than through external influence or instruction \cite{willshawSO}. Complex dynamic systems which have many and diverse elements interacting with each other, may display features of self-organization. This phenomena can be triggered by seemingly random fluctuations, amplified by positive feedback.\par 
Fundamental characteristics of this organization are:
\begin{enumerate}
	\item {\bf decentralization} : the organization is distributed over all the components of the system
	\item {\bf robustness} : the system is able to survive or self-repair perturbations
\end{enumerate}
{\bf "In chaos theory, self-organizations represent "islands" of predictability in a sea of chaotic unpredictability".}


\subsection{Self-Organized Criticality}
Self-organized criticality (SOC) has been a controversial and complicated topic to properly define in literature. Authors considered different definitions for SOC.


In his 1996 paper "Simplest Possible Self-Organized Critical System" \cite{simplestSOC}, Flyjberg explains that a SOC system is a driven dissipative system consisting of:
\begin{enumerate}
	\item a {\it medium} which has:
	\item {\it disturbances} propagating through it, causing
	\item a {\it modification} of the medium, such that eventually
	\item the medium is in a {\it critical state} and
	\item the medium is {\it modified no more}
\end{enumerate}


Per Bak, Chao Tang and Kurt Wiesenfeld \cite{BTW} formulated the followinng key points when they defined Self-Organized Criticality (SOC):
\begin{itemize}
	\item Spatial and temporal scaling must usually be unavoidably connected.
	\item In contrast to phase transitions (or chaos) seen as a certain fixed point in the order parameter space, there must be a robust, widespread new kind of spatio-temporal critical behaviour which arises from self-organization.
	\item The conditions for SOC to be seen in a system are slow driven interaction and existence of a threshold. Also, dissipation has a crucial role in maintaining this kind of state.
	\item Spacetime fractals are snapshots of the SOC state.
\end{itemize}\par
In SOC, criticality has a particular consequence regarding continuous transitions. At the critical point, correlations become long ranged (meaning they follow a power law) and, equivalently, fluctuations occur on all length scale, meaning that the fluctuations also display a power law dependence, with non-zero exponent.\

\vspace{5mm}
In a 2015 comprehensive review of 25 years of Self-Organized Criticality \cite{Watkins2015}, the authors argued further points that help define this concept:

\paragraph{The necessary conditions to observe a SOC state:}
\begin{enumerate}
	\item Non trivial scaling.
	\item Spatio-temporal power law correlations.
	\item Apparent self tuning to the critical point.
\end{enumerate}\par 

\paragraph{The sufficient conditions to generate a SOC state:}
\begin{enumerate}
	\item Non-linear interaction, usually as thresholds.
	\item Avalanching (intermittently, expected when the system is slow driven and has thresholds).
	\item Separation of time scale (to sustain avalanches).
\end{enumerate}

\paragraph{Importance in seismicity modeling} \par 

In the second chapter of this report, the main models of seismicity are presented, which posses a great deal of insight into the possible mechanism of earthquake generation and especially the self-organized criticality state that a seismic zone may reach before triggering an avalanche of earthquakes, comprising of a big event, followed by its aftershocks.



%--------------------------------------------------------
\section{Network Theory}
As mentioned earlier, a great tool of studying complex systems is through Network Theory \cite{latora}. Initially born out of graph theory from mathematics, it has become a great instrument in analyzing complex interacting systems by being able to keep track of their parts and the relationships between them in a simple and intuitive manner. Complex networks are networks whose structure is irregular, complex and dynamically evolving in time.\par 

\subsection{Historical View on Networks}

\paragraph{The six degrees of separatiom}
The concept of networking has been born much earlier than one would think \cite{linked}. Although it has been mathematically defined in recent history, the most common network one turns to when speaking of this subject, the social network we all are a part of, has been brilliantly identified in 1967 by Stanley Milgram, a Harvard professor, who made a groundbreaking study on our interconnectivity.\par 
Milgram's goal was to find the average "distance" between two random people in the United States, i.e. how many acquaintances would it take to link two randomly selected persons. He first chose a target person (for example, a broker in boston), then sent letters to random people across the United States with the target's picture and information. When a person receives the letter, they first have to record their name on a list in the letter to keep track of the path. Then send the letter to the target only if they knew them on a first name basis, or to a person that they think would be "closer" to the targets. The process continues until the target is reached. The result: 42 of 160 letters reached the targets, and the average distance was 5.5, thus the famous "six degrees of separation".\par 
The discovery was intriguing because it suggests that despite the vast size of our social network, we live in a world in which no one is more than a few handshakes away from anyone else. Our society is a small world because it is a very dense web.

\begin{figure}[!h]
  \centering
  \includegraphics[width=.5\linewidth]{smallworld}
  \caption{Stanley Milgram's Small Worlds Experiment - image from \cite{smallworlds}: A letter randomly sent to a citizen in Nebraska starts on a 6 person's journey to it's target in Boston. Each person mailed the letter to an acquaintance that they thought would be closer to the target. The second to last person, mails the letter to the target because they knew him personally.}
  \label{fig:smallWorld}
\end{figure}



\paragraph{Euler's bridges}
Even earlier than Milgram's experiment, the problem that is thought to be the birth of graph theory is the Seven Bridges of Königsberg \cite{eulerBridges}.\par
The problem is formulated as follows: is there a possible walk through the city that would cross each of the seven bridges once and only once (the configuration of land-bridges is depicted below)? "No" deduces Leonhard Euler in 1736 and gives it's solution, along with other representations which many consider today the first conjuncture of graph theory in mathematics.
\begin{figure}[!h]
  \centering
  \includegraphics[width=.8\linewidth]{eulerBridges}
  \caption{A depiction of what the problem looks like then and now. On the left, a map of Seventeenth-centuri Königsberg with the bridges in question highlighted. In the middle, a visualization of how Euler graphically represented the problem and on the right, how we represent the issue today, in modern graph theory. Image Credit: Bogdan Giușcă, Wikipedia}
  \label{fig:eulerBridges}
\end{figure}

Euler's solution came from the observation that by walking in the graph, except for the start and finish nodes, one must enter a node as many times as he exists, so the nodes must be touched by even numbers of bridges. This made the connection between walks and node degrees in a graph, meaning that the necessary and sufficient condition for the desired walk is that the graph may have exactly zero or two nodes of odd degree. Since in the problem, all land masses are connected by odd number of bridges, the proposed walk is impossible.\par 
Euler's contribution was two-fold: firstly, it is considered to be the first theorem in graph theory and the theory of networks and secondly, the realisation that the information in the problem was the number of bridges and their endpoints represented the beginning of the development of a new area of mathematics named topology.



\subsection{Graph Theory} 
Today, networks have applications in a vast number of domains: from sociology, economics, finance to statistical physics, electrical engineering, climatology and neuroscience. Graph theory is the framework for the exact mathematical treatment of complex networks. An undirected (or directed) graph $G=(\mathcal{N},\mathcal{L})$ consists of two sets $\mathcal{N}$ such that $\mathcal{N} \neq \emptyset$ and $\mathcal{L}$ represents the set of unordered (or ordered) pairs of elements of $\mathcal{N}$:
\begin{itemize}
	\item $\mathcal{N} \equiv \{ n_1,n_2,...,n_N \}$ = the nodes (or vertices, points) of G
	\item $\mathcal{L} \equiv \{ l_1,l_2,...,l_K \}$ = the links (or edges, lines) of G 
\end{itemize}
$G(N,K) = (\mathcal{N},\mathcal{L})$ represents a graph with $N$ nodes and $K$ edges.\par 
The nodes are reffered to by their indices, their order $i$ in the set $\mathcal{N}$ and the edges, by the two indices of the nodes $(i,j)$ they join together. The usual graphical representation is composed of dots for the nodes and lines between the nodes which are connected through an edge. If the graph is directed, then the line transforms into an arrow. The links can also be weighted, meaning there are more connections between the same nodes and this would be graphically represented by making these lines thicker.

\begin{figure}[!h]
  \centering
  \includegraphics[width=.7\linewidth]{graphs}
  \caption{Graphical representation of a few types of graphs - image from \cite{latora}, each with $N=7$ nodes and $K=14$ edges: (a) undirected, (b) directed, the arrows showing the direction of each link and (c) weighted undirected, each link's weight $W_{i,j}$ reported on the respective line.}
  \label{fig:graphs}
\end{figure}

A few fundamental elements of a graph:
\begin{itemize}
	\item a {\it walk} fom node $i$ to node $j$ = alternating sequence of nodes and edges that begins in $i$ and ends in $j$;
	\item a {\it trail} = a walk in which no edge is repeated;
	\item a {\it path} = a walk in which no node is repeated;
	\item the {\it shortest path} = the walk of minimal length between two nodes;
	\item a {\it cycle} = a closed walk of at least three nodes, usually defined by it's length $k$ or $C_k$ (example: $C_3$ is a triangle, $C_4$ is a quadrilater).
\end{itemize}

Finally a graph is mathematically described by it's {\it adjacency matrix} $\mathcal{A}$, a $N \times N$ matrix whose entries $a_{ij}$ are either $1$ if link $l_{ij}$ exists or $0$ 
otherwise.\par 

The properties and characteristics of the graphs that interests us will be defined when we construct and analyze seismic networks later in Chapter \ref{chap:quakeNetwork}.

