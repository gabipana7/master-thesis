\addcontentsline{toc}{chapter}{Introduction}
\chapter*{Introduction}

In this work we wish to analyze various seismic regions around the globe from the perspective of complex networks. We expect that this would help give some insight into the mechanism of large events creation, avalanches of aftershocks and spatio-temporal span of these events and it would help us prove that earthquakes behave as a complex, self-organized critical system.\par 
This report is organized in five chapters: \par 
Chapter \ref{chap:complexSystems} is dedicated to familiarizing with terms such as complex systems, critical systems, self-organization and self-organization to criticality. A short history and introduction in network theory and graph theory is also presented. \par 
Chapter \ref{chap:quakeModels} deals with the most known system that shows self-organization to criticality behaviour, the classical sandpile model. The simplified and the original model help us understand this peculiar characteristic that appears in some complex systems in nature. Also, a staple model of seismicity is presented in this chapter: the Olami-Feder-Christensen slider-blocks model. \par 
Chapter \ref{chap:quakeDatabase} is dedicated to presenting the databases from which we extract information about earthquakes and how we manipulate these databeses in our analysis. Also, we present how we split a seismic region into small cubes, which will become the nodes of our seismic network.\par
Chapter \ref{chap:quakeNetwork} is the main focus of our work. Here we explain how we construct our seismic network and the centrality measures we use to analyze it. We present results regarding distribution of connectivity, weighted and unweighted. We define the building blocks of networks - graph motifs and we apply motif discovery tools to extract 3 and 4 nodes motifs from our seismic networks. Distributions of surfaces and volumes of these motifs, weighted by mean and total magnitude are presented. Finally, suggestive visualization of the real networks created, for a small part of the network are introduced. \par 
Chapter \ref{chap:quakeCorrelations}, the final one of our report, deals with spatio-temporal autocorrelations in our seismic networks. The goal is to give an undertanding into the seemingly chaotic behaviour of earthquakes.\par 
In the end, we conclude the report with a summary of results and insights obtained that point towards the self-organized to criticality nature of seismic zones.


