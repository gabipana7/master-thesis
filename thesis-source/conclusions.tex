\addcontentsline{toc}{chapter}{Conclusions}
\chapter*{Conclusions}
In conclusion, we have shown how to access and extract information about earthquakes from the databases available. We presented how to construct various seismic networks for regions around the globe and by measures of connectivity and motif discovery we proved that the scale-free nature of these networks is {\bf very robust}.\par 
With the help of visualization tools such as VTK and ParaView, we have shown how these network can be graphically represented in real, geographic coordinate space, with the connectivity, edge weights and motifs pictured with the help of graphs.\par 
We also computed spatial and temporal autocorrelation functions for the regions analyzed and concluded two distinct behaviours when taking into account lower magnitude, in contrast with higher magnitude earthquakes. At lower energies, events can be viewed as some kind of chaotic fluctuations, while at higher energies, some pattern may emerge. \par 
This work sustains the popular notion that earthquakes are complex systems that show self-organization to criticality behaviour. \par 
Further study can be done in this space, with the possibility of analyzing other seismic regions by employing the same computational tools used in this report. Also, from the perspective of network theory, other centrality or community measures may give insight about clusters of earthquakes, opening the possibility of splitting a region into more relevant sections. 