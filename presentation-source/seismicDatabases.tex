\section{Seismic Databases}


%----------------------------------------------------------------%
\begin{frame}{Seismic Databases}
For each seismic region we wish to study, the first step is to collect the earthquakes databases available online, published by the respective region institute:
\begin{itemize}
	\item {\bf Vrancea(Romania)} - National Institute for Earth Phyiscs.
	\item {\bf California(USA)} - Southern California Earthquake Center.
	\item {\bf Italy} - National Institute for Geophysics and Vulcanology.
	\item {\bf Japan} - Japan Meteorological Agency.
\end{itemize}
\end{frame}


%----------------------------------------------------------------%
\begin{frame}{All data available in databases}
\begin{center}
\centering
 \begin{tabular}{ |c||c|c|c|c|  }

 \hline
 \multicolumn{5}{|c|}{Seismic Databases} \\
 \hline
 Seismic Zone & Timeframe & Latitude & Longitude & Depth\\
 \hline
 \hline
 \multirow{2}{8em}{Romania} & 0984-01-01 & 43.594$^{\circ}$N & 20.1$^{\circ}$E & 0\\
 & 2021-02-28 & 48.23$^{\circ}$N & 26.14$^{\circ}$E & 218.4\\
 \hline
 \multirow{2}{8em}{California(USA)} & 1932-01-02 & 32$^{\circ}$N & -114$^{\circ}$W & 0\\
 & 2020-12-31 & 37$^{\circ}$N & -122$^{\circ}$W & 51.1\\
 \hline
 \multirow{2}{8em}{Italy} & 1986-01-01 & 30.61$^{\circ}$N & -6.08$^{\circ}$W & 0\\
 & 2020-12-31 & 47.998$^{\circ}$N & 36.02$^{\circ}$E & 644.4\\
 \hline
 \multirow{2}{8em}{Japan} & 1919-01-11 & 17.41$^{\circ}$N & 114.78$^{\circ}$E & 0\\
 & 2019-08-31 & 54.97$^{\circ}$N & 160.17$^{\circ}$E & 698.4\\
 \hline
 \end{tabular}
\end{center}
\end{frame}


%----------------------------------------------------------------%
\begin{frame}{Romania and Vrancea} 

\begin{figure}[!h]
\begin{subfigure}{.5\textwidth}
  \centering
  \includegraphics[width=.9\linewidth]{quakesRomania_map}
  \caption{Romania: 29186 earthquakes,\\ from 1976-02-03 13:29:16 to 2021-02-28 16:57:29.}
  \label{fig:sfigRo}
\end{subfigure}%
\begin{subfigure}{.5\textwidth}
  \centering
  \includegraphics[width=.9\linewidth]{quakesVrancea_map}
  \caption{Vrancea: 7512 earthquakes,\\ from 1976-08-19 19:03:01 to 2021-02-28 00:11:55.}
  \label{fig:sfigVrancea}
\end{subfigure}
\caption{Earthquakes distribution for Romania (left) and Vrancea (right) seismic zones with magnitude $>1$.}
\label{fig:simpleScatterRoVr}
\end{figure}

\end{frame}


%----------------------------------------------------------------%
\begin{frame}{California(USA)}
\begin{figure}[!h]
\centering
\includegraphics[width=.5\linewidth]{quakesCalifornia_map}
\caption{Earthquakes distribution for California(USA), seismic zone: 221113 events with magnitude $>1$, from 1984-01-01 18:27:55 to 2020-12-31 23:04:53.}
\label{fig:simpleScatteritaly}
\end{figure}
\end{frame}



%----------------------------------------------------------------%
\begin{frame}{Italy}

\begin{figure}[!h]
\centering
\includegraphics[width=.5\linewidth]{quakesItaly_map}
\caption{Earthquakes distribution for Italy seismic zone: 319567 events with magnitude $>1$, from 1986-01-01 17:22:53 to 2020-12-31 23:41:18.}
\label{fig:simpleScatteritaly}
\end{figure}

\end{frame}


%----------------------------------------------------------------%
\begin{frame}{Japan}
\begin{figure}[!h]
\centering
\includegraphics[width=.5\linewidth]{quakesJapan_map}
\caption{Earthquakes distribution for Japan seismic zone: 595713 events with magnitude $>2$, from 1992-01-01 00:54:03 to 2019-08-31 23:54:24.}
\label{fig:simpleScatterJapan}
\end{figure}
\end{frame}


%----------------------------------------------------------------%
\begin{frame}{Earthquake Table}
The fundamental tool we use in our analysis is the Earthquakes Table:

\begin{figure}[!h]

\centering
\includegraphics[width=\linewidth]{quakesTable}
\caption{The Earthquakes Table - our fundamental tool for the following analysis, containing the basic information from the databases available online ({\it date, latitude, longitude, depth and magnitude}) and our computations: the $energyRelease$ and the "cube parametrization" with indexing both in the "cubes space": $x$, $y$, $z$ and the real space: $cubeLatidue$, $cubeLongitude$, $cubeDepth$.\\ }
\label{fig:quakesTable}
\end{figure}
The data in this example represents the first 5 events in the Vrancea seismic zone, starting with the year 1976, as a pandas DataFrame in Python.

\end{frame}

%----------------------------------------------------------------%
\begin{frame}
\begin{enumerate}
	\item {\bf energyRelease:} we can roughly estimate the energy release of each event by converting the moment magnitude $M_W$ to energy using the equation $log E = 5.24 + 1.44M$ where $M$ is the magnitude. 
	\item {\bf Cube splitting:} in order to build our earthquakes network, we need to divide the spatial region selected into cubes and place each event in it's respective cube.
\end{enumerate}

\begin{figure}[h!]
\centering
\tikz[remember
picture]{\node(1BL){\includegraphics[width=4cm]{quakesVrancea_grid}};}%
\hspace*{3cm}%
\tikz[remember picture]{\node(1BR){\includegraphics[width=4cm]{quakesVrancea_cubeSplit}};}
\end{figure}
\tikz[overlay,remember picture]{\draw[-latex,thick] (1BL) -- (1BL-|1BR.west)
node[midway,below,text width=2cm]{Cube split};} 

\end{frame}

