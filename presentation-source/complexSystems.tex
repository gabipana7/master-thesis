\section{Complex Systems}

%{
%\usebackgroundtemplate{%
%\tikz\node[opacity=0.2] {\includegraphics[height=\paperheight,width=\paperwidth]{shoreline}};}
%----------------------------------------------------------------%
\begin{frame}{Complex Systems}

%{\bf Diversity} applies to a collection of entities or a population; it requires a multitude of objects that are to be analyzed. For example, cities, ecosystems, molecular matter, plasma are diverse. When speaking of diversity, we can mean a few different characteristics of a population:
%\begin{itemize}
%	\item {\it variation} of some attribute (eg. difference of height in a group of people).
%	\item {\it diversity} of types (eg. different ethnicities in a social group).
%	\item differences in {\it configuration} (eg. different arrangments in housing units of a group of people).
%\end{itemize} 

{\bf Complexity} can be thought of as particular structures and patterns that cannot be easily described or predicted. A system becomes complex when {\it diverse} rule-following entities behave in an {\it interdependent} way. These entities interact over a {\it contact structure} or {\it network}. Characteristics:
\begin{itemize}
	\item {\it adaptation} (eg. in a social system, individuals can learn, or in an ecosystem natural selection can take place)
	\item {\it robustness} (eg. they exhibit a certain behaviour at any spatio-temporal scale)
	\item {\it large events occurence} (eg. large earthquakes)
	\item {\it equillibrium states, fixed points, patterns or chaotic behaviour}
\end{itemize}

\end{frame}
%}


%----------------------------------------------------------------%
\begin{frame}{Stanley Milgram's Small Worlds}
\begin{figure}[!h]
  \centering
  \includegraphics[width=.5\linewidth]{smallworld}
  \caption{Stanley Milgram's Small Worlds Experiment: A letter randomly sent to a citizen in Nebraska starts on a 6 person's journey to it's target in Boston. Each person mailed the letter to an acquaintance that they thought would be closer to the target. The second to last person, mails the letter to the target because they knew him personally. Image from Elisa Baek et. al., Social Network Analysis for Social Neuroscientists”}
  \label{fig:smallWorld}
\end{figure}
\end{frame}


%----------------------------------------------------------------%
\begin{frame}{Euler's bridges}
Even earlier than Milgram's experiment, the problem that is thought to be the birth of graph theory is the Seven Bridges of Königsberg.\par

\begin{block}{Problem}
Is there a possible walk through the city that would cross each of the seven bridges once and only once?\par 
"No" - Leonhard Euler, 1736
\end{block}

\begin{figure}[!h]
  \centering
  \includegraphics[width=.9\linewidth]{eulerBridges}
  \caption{A depiction of what the problem looks like then and now. On the left, a map of Seventeenth-century Königsberg with the bridges in question highlighted. In the middle, a visualization of how Euler graphically represented the problem and on the right, how we represent the issue today, in modern graph theory. Image credit: Bogdan Giușgă, Wikipedia}
  \label{fig:eulerBridges}
\end{figure}
\end{frame}


%----------------------------------------------------------------%
\begin{frame}
\begin{columns}
          \column{0.28\linewidth}
             \centering
             \includegraphics[width=3.5cm]{eulerBridgeGraph}
           \column{0.68\linewidth}
              \textbf{Euler's Solution}
by walking in the graph, except for the start and finish nodes, one must enter a node as many times as he exists, so the nodes must be touched by even numbers of bridges. This made the connection between walks and node degrees in a graph, meaning that the necessary and sufficient condition for the desired walk is that the graph may have exactly zero or two nodes of odd degree.
         \end{columns} 
         
\vspace{5mm}
         
Since in the problem, all land masses are connected by odd number of bridges, the proposed walk is impossible.\par 

\vspace{5mm}

%Euler's contribution was two-fold: firstly, it is considered to be the first theorem in graph theory and the theory of networks and secondly, the realisation that the information in the problem was the number of bridges and their endpoints represented the beginning of the development of a new area of mathematics named topology.
\end{frame}


%----------------------------------------------------------------%
\begin{frame}{Graph Theory}
Graph theory is the framework for the exact mathematical treatment of complex networks. An undirected (or directed) graph $G=(\mathcal{N},\mathcal{L})$ consists of two sets $\mathcal{N}$ and $\mathcal{L}$:
\begin{itemize}
	\item $\mathcal{N} \equiv \{ n_1,n_2,...,n_N \}$ = the nodes (or vertices, points) of G
	\item $\mathcal{L} \equiv \{ l_1,l_2,...,l_K \}$ = the links (or edges, lines) of G 
\end{itemize}
$G(N,K) = (\mathcal{N},\mathcal{L})$ represents a graph with $N$ nodes and $K$ edges.\par 

\begin{figure}[!h]
  \centering
  \includegraphics[width=.65\linewidth]{graphs}
  \caption{Graphical representation of a few types of graphs, each with $N=7$ nodes and $K=14$ edges: (a) undirected, (b) directed, the arrows showing the direction of each link and (c) weighted undirected, each link's weight $W_{i,j}$ reported on the respective line. Image credit: S. Boccaletti et. al., “Complex networks: Structure and dynamics”}
  \label{fig:graphs}
\end{figure}

\end{frame}

\begin{comment}
%----------------------------------------------------------------%
\begin{frame}
\begin{figure}[!h]
  \centering
  \includegraphics[width=.8\linewidth]{graphs}
  \caption{A few types of graphs, each with $N=7$ nodes and $K=14$ edges: (a) undirected, (b) directed, the arrows showing the direction of each link and (c) weighted undirected, each link's weight $W_{i,j}$ reported on the respective line.}
  \label{fig:graphs}
\end{figure}
\end{frame}


%----------------------------------------------------------------%
\begin{frame}
A few fundamental elements of a graph:
\begin{itemize}
	\item a {\it walk} fom node $i$ to node $j$ = alternating sequence of nodes and edges that begins in $i$ and ends in $j$;
	\item a {\it trail} = a walk in which no edge is repeated;
	\item a {\it path} = a walk in which no node is repeated;
	\item the {\it shortest path} = the walk of minimal length between two nodes;
	\item a {\it cycle} = a closed walk of at least three nodes, usually defined by it's length $k$ or $C_k$ (example: $C_3$ is a triangle, $C_4$ is a quadrilater).
\end{itemize}

Finally a graph is mathematically described by it's {\it adjacency matrix} $\mathcal{A}$, a $N \times N$ matrix whose entries $a_{ij}$ are either $1$ if link $l_{ij}$ exists or $0$ 
otherwise.\par 

\end{frame}

\end{comment}



%\usebackgroundtemplate{%
%\tikz\node[opacity=0.15] {\includegraphics[height=\paperheight,width=\paperwidth]{criticality}};}
%----------------------------------------------------------------%
\begin{frame}{Criticality and Self-organization}

{\bf Critical phenomena} refers to peculiar behaviour of a system when it is near or at the point of a continuous-phase transition, also called a {\it critical} point, which is a point at which the system changes from one state to another without a jump or discontinuity in it's properties such as internal energy, density or magnetization.

\vspace{5mm}

{\bf Self-organization} is the process by which individual components of a system organize their communal behaviour to create global order by interactions amongst themselves rather than through external influence or instruction. Complex dynamic systems which have many and diverse elements interacting with each other, may display features of self-organization. \par
%This phenomena can be triggered by seemingly random fluctuations, amplified by positive feedback.\par 
Fundamental characteristics of this organization are:
\begin{enumerate}
	\item {\bf decentralization} : the organization is distributed over all the components of the system
	\item {\bf robustness} : the system is able to survive or self-repair perturbations
\end{enumerate}
{\bf "In chaos theory, self-organizations represent "islands" of predictability in a sea of chaotic unpredictability".}

\end{frame}




%\usebackgroundtemplate{%
%\tikz\node[opacity=0.35] {\includegraphics[height=\paperheight,width=\paperwidth]{soc}};}
%----------------------------------------------------------------%
\begin{frame}{Self-Organized Criticality}
In his 1996 paper "Simplest Possible Self-Organized Critical System", Flyjberg explains that a SOC system is a driven dissipative system consisting of:
\begin{enumerate}
	\item a {\it medium} which has:
	\item {\it disturbances} propagating through it, causing
	\item a {\it modification} of the medium, such that eventually
	\item the medium is in a {\it critical state} and
	\item the medium is {\it modified no more}
\end{enumerate}
\end{frame}




%\usebackgroundtemplate{%
%\tikz\node[opacity=0.2] {\includegraphics[height=\paperheight,width=\paperwidth]{fractal}};}
%----------------------------------------------------------------%
\begin{frame}
Per Bak, Chao Tang and Kurt Wiesenfeld formulated the followinng key points when they defined Self-Organized Criticality (SOC):
\begin{itemize}
	\item Spatial and temporal scaling must usually be unavoidably connected.
	\item There must be a robust, widespread spatio-temporal critical behaviour which arises from self-organization.
	\item Slow driven interaction and existence of a threshold.
	\item Dissipation has a role in maintaining a SOC state
	%\item Spacetime fractals are snapshots of the SOC state.
\end{itemize}
\end{frame}

